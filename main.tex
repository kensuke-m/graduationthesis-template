\documentclass[a4paper,twocolumn,10pt]{ltjsarticle}

\usepackage{geometry}
\geometry{top=20mm, bottom=24mm, left=23mm, right=23mm}

\usepackage{graphicx}

\makeatletter
  % maketitle
  \def\@maketitle{%
    \newpage
    \begin{center}%
     %和文題名
     {\Large \textbf{\@title} \par}%
     \vskip 9pt
     %和文著者
     {\large \lineskip .0em
     \begin{tabular}[t]{c}%
      \@author
     \end{tabular} \par}%
     %英文題名
     \vskip 1.5em
     {\Large \textbf{\etitle@etitle} \par}%
     \vskip 0.5em
     %英文著者
     {\large \lineskip .0em
     \begin{tabular}[t]{c}%
      \etitle@eauthor
     \end{tabular} \par}%
     \vskip 0.5em
    \end{center}%
    \par\vskip 1.5em
    \ifvoid\@abstractbox\else\centerline{\box\@abstractbox}\vskip1.5em\fi
    }
  % \title \author は既存のマクロを利用 \@title \author に値がセットされる.
  \newcommand{\etitle@etitle}{}
  \newcommand{\etitle@eauthor}{}
  \newcommand{\affiliation}[1]{\renewcommand{\etitle@affiliation}{#1}}
  \newcommand{\etitle}[1]{\renewcommand{\etitle@etitle}{#1}}
  \newcommand{\eauthor}[1]{\renewcommand{\etitle@eauthor}{#1}}
\makeatother

% !!!重要!!! ファイル先頭からこの行までの間は改変しないこと.

% 必要に応じて適宜パッケージを追加
% 例:特殊な数学記号を利用する→ \usepackage{amsmath} など
\usepackage{graphicx}  % 図をインポートするパッケージ
\usepackage{url}       % URLを \url{} で括ると見やすくなる

% 和文題名
\title{宮下ゼミ卒業論文テンプレート2023}
% 和文著者
\author{京都女子大学 現代社会学部 現代社会学科 宮下ゼミ\\
20241000 西本 願子}
% 英文題名
\etitle{Manuscript Template for Graduation Thesis in Miyashita Lab 2023}
% 英文著者. 和文所属と同様です
\eauthor{Miyashita Lab., Faculty for the Contemporary Society, Kyoto Women's University\\
20241000 Nishimoto Ganko}

% 論文の概要
\begin{abstract}
本稿は宮下ゼミの学生向けに書かれた卒業論文のテンプレートであり,同時に卒論執筆の際の注意点などをまとめたものでもある.
主に\LaTeXの使い方に関する説明と,論文の内容に関して``するべきこと''と``するべきでないこと''をまとめている.
本稿自体も\LaTeXを利用して執筆されているので,卒論作成の参考になれば幸いである.
さて,本来この部分には卒業論文の概要(論文内容を簡潔にまとめたもの)を書く.
つまり研究の動機や背景,問題,解決手法,結論などを簡潔にまとめた文章を「概要」としてまとめる.
この概要を読めば論文内容がだいたい示されているようにすることが重要である(もともと「概要」とはそのようなものである).
文章量は特に定めないが,だいたい4〜8行程度にすればバランスがよいと思われる.
\end{abstract}

\begin{document}
% \begin{document} 直後に \maketitle コマンドを置く
\maketitle

\section{はじめに}

これは京都女子大学現代社会学部宮下ゼミの学生向けに書かれた卒業論文テンプレートである\footnote{本稿の最新版は\url{https://github.com/kensuke-m/graduationthesis-template}から入手できる.}.
宮下ゼミの学生はこの内容をよく理解し,すばらしい卒業論文を執筆してもらいたい.
また,このテンプレートに足りないところや誤りがあればいつでも指摘してほしい.

宮下ゼミの卒業論文はA4縦型2段組を採用している.
これは情報処理学会論文誌や研究報告と同様であり,演習Vで論文紹介をしたときに見慣れた体裁かと思われる.

本稿ではまず\LaTeX のスタイルファイルを用いた論文のフォーマットについて述べる.
論文フォーマットについて詳しくは\ref{sec:format}節で後述する指針を参照してほしい.
また,そこで説明されていることの外は標準的な\LaTeXのコマンドや他のスタイルファイルなどを利用して構わない.
本稿は卒業論文のテンプレートとして実際に\LaTeX を利用しているので,論文執筆時にはソースファイル({\tt main.tex})などを参考にされたい.

また,卒業論文を執筆する際に著者(これを読んでいるあなたである)がするべきこと,するべきでないこともまとめてある.
本稿の後半ではこれらについて指針を述べているので,卒論提出前はもとより執筆中にもぜひ継続的にチェックしていただきたい.

明確で簡潔に書かれた文書に勝るものはない.
第2次大戦中,イギリスのウィンストン・チャーチル首相は「我が国の外務省は長い報告書を書けばナチスに勝てると思っているようだ」としばしばこぼしたという\cite{newsweek}.
卒業論文に限らず他人に読んでもらうことが目的の文書は,できる限り短く,論旨が明確で,理解しやすいものを目指そう.

\section{卒業論文執筆の流れ}

卒業論文は以下のように執筆し,提出することが望ましい.

\subsection{準備}

この卒業論文テンプレート一式は\\
\url{https://github.com/kensuke-m/graduationthesis-template}\\
から入手できる.
ここには以下のファイルが含まれる.
\begin{Enumerate}
 \item \|reference.bib |: 参考文献リストのサンプル
 \item \|graphs.pdf    |: 図のサンプル
 \item \|latexmkrc     |: \LaTeX の設定ファイル
 \item \|main.tex      |: 本稿のソースファイル
 \item \|README.md     |: 利用法の説明
\end{Enumerate}
これらのファイルの利用法は {\tt README.md} に書かれているので,まずはそれにしたがって卒論執筆の準備をしよう.

卒業論文はこのファイルのように文書処理システム \LaTeX を利用して作成し,PDF形式で提出する.
以前は \LaTeX を各自がPCにインストールして利用していたが,近年ではWWW上のアプリケーションとして利用するOverleaf\cite{refwww}のようなシステムが普及しているので,それを使う.

\subsection{執筆}

本稿にしたがって卒業論文を執筆し,\LaTeX ソースファイルからPDFファイルを作成する.
前述した Overleaf では「リコンパイル」ボタンのクリックまたは Alt キーを押しながら Return キーを押下することでPDFファイルが作成され,WWWブラウザ内でプレビューできる.
その際エラーが発生すれば適宜修正し,プレビューできた場合は自分の意図通りの体裁になっているか,誤字や脱字などはないかなどをチェックする.
そして説明の足りないところを補ったり,図表を挿入してわかりやすくしたりなどの改善を繰り返しつつ,卒論を完成させる.

ある程度執筆が進んだらPDFファイルをダウンロードして紙に印刷してみることをお勧めする.
PCのディスプレイやタブレットなどの透過光で読む原稿と紙のように反射光で読む原稿では見え方が異なり,一方では気付かなかったことに他方で気付くことが多い.
特に完成した卒論を提出する直前には印刷したものを一度通して読んでみると良い.

自分ではこれ以上改善すべき箇所が見当たらないというところまで来たら,教員に提出してコメントをもらおう.
自分では気付かなかったことが必ずあるはずで,教員から何か所も指摘されるはずである.
それらの指摘を活かして卒論を改善してもらいたい.

\subsection{提出}

宮下ゼミでは卒業論文を提出する際に,ゼミ内での提出と教務課への提出の2つが要求される.
もちろん両方が必要である.

前者では,PDF形式のファイルを教員が指定した方法で然るべき場所へアップロードする.
具体的には,Microsoft Teams内のその年度の演習VIチャネルに「卒業論文」という課題が作成されるので,そこへPDFファイルをアップロードする.

後者は,印刷された原稿(左上をホチキス等で綴じること)を教務課窓口へ提出するのが通例である\footnote{社会情勢の変化により提出方法が変更される可能性があるので京女ポータルのお知らせに注意すること.}.
このとき,卒業論文題目用紙という紙を添付する必要があり,これは京女ポータル上で卒業論文のタイトルや指導教員名などを入力し,印刷したものに指導教員が押印する.
時期が来たら研究室の扉にシャチハタ印がぶら下げられるので各自で押印のこと.

どちらも2024年1月15日(月)17時が〆切である.
〆切際の教務課窓口は混雑が予想されるので余裕を持って提出すると良い\footnote{前年12月後半から受け付けてもらえるようである.}.

\section{卒業論文フォーマット}
\label{sec:format}

卒業論文は電子的な媒体で読む(例えばPCやタブレットの画面で閲覧する)のに適するよう作成し,PDF形式のファイルで提出する.
以下では卒業論文のフォーマットについて指針を述べるので,これにしたがって執筆すること.
\LaTeX を利用した文書作成について一般的な知識は \cite{okumura2020} を参考にされたい.

\subsection{卒業論文の構成}

卒業論文の \LaTeX ソースファイル({\tt main.tex})


\subsection{論文の書式}

読みやすく明瞭かつ正確な原稿とするために,主に以下の指示にしたがって執筆する.

\begin{itemize}
 \item 原稿はA4判2段組で8ページ以下とする.
 \item ページ数は,わかりやすさを損ねない範囲でできるだけ少なくする.
 \item 長大なプログラムリストなどが必要不可欠な場合は,付録として原稿の末尾に付す(これはページ数に含めない).
 \item 先頭ページ上部に題名,所属,著者名を日本語と英語で記載する.
 \item 本文の文字の大きさは10ポイントとする(図表中の文字も原則として本文と同程度の大きさとする).
 \item 本文は日本語または英語で記述する.
 \item 本文は読みやすいようにいくつかの節\footnote{大抵の場合,「はじめに」で始まり,背景や先行研究,本論などを述べ,「おわりに」で終わる.その後,謝辞と参考文献が続く.この原稿もそれに準じている.}や段落に分け,明瞭簡潔な記述を心がける.
 \item 本文の文章は常体で記述し,句読点はコンマ「,」とピリオド「.」(どちらも全角文字)とする.
 \item 著者自身を表したいときは「著者は〜」などとし,「私」は用いない.
 \item 客観的な事実に基づく記述を心がけ,不用意に根拠のない主張など表明しない.
 \item 文章や図表などを引用したり参考にしたりする際は出典を明示し,著作権を侵害しないよう注意する.
 \item お世話になった人には謝辞を述べる.
\end{itemize}

\subsection{PDFファイルの作成}

卒業論文として提出できるファイル形式はPDFだけである.
その際,以下のことに注意する.

\begin{itemize}
 \item ファイル名には拡張子({\tt .pdf})を必ず付すこと(ファイル名そのものは何でも構わない).
 \item 図として画像を利用する場合は解像度を適切に設定する(少なくとも150dpi程度以上).
 \item 編集不可・印刷不可などのセキュリティ設定を施さない.
\end{itemize}

\section{図表の書き方}

図表は本文中に埋め込み,番号とキャプションを付す.
図表の配置と番号付けは\LaTeX{}により自動的に行われるのでそれに任せる.
キャプションはその図表を表す端的なものとする.
図表は必ず本文中から参照するものとし,その際は図表の番号で参照する.

以下,図の例を図\ref{fig:graphs}に示す.
図は中央寄せ(センタリング)とし,図のキャプションは図の直下に記述する.

\begin{figure}[htb]
 \begin{center}
  \includegraphics{graphs.pdf}
  \caption{2部グラフから1モードグラフへの変換}
  \label{fig:graphs}
 \end{center}
\end{figure}

また,表の例を表\ref{tab:competency}に示す.
表もセンタリングし,表のキャプションは表の直上に記述する.

\begin{table}[htb]
 \begin{center}
  \caption{コンピテンシーディクショナリの項目数}
  \label{tab:competency}
  \begin{tabular}{cccc}
   \hline
   名称 & タスク & スキル & 関連知識\\
   \hline
   項目数 & 639 & 491 & 8843\\
   分類 & 203 & 150 & 3308\\
   \hline
  \end{tabular}
 \end{center}
\end{table}

\section{参考文献の書き方}

参考文献は論文での出現順に記載し,巻末に一覧を掲載する.
その際,雑誌の場合\cite{newsweek,refjournal1,refjournal2}と書籍の場合\cite{refbook1,refbook2},WWWページの場合\cite{refwww,readme}それぞれ以下の通りに記述し,この文章のように参考文献を参照している箇所に参考文献番号を記載する.

\begin{itemize}
 \item 雑誌の場合:著者名,タイトル,雑誌名,巻,号,ページ,発行年.
 \item 書籍の場合:著者名,書名,参照ページ(あれば),発行所,発行年.
 \item WWWの場合:タイトル,URL,アクセス日.
\end{itemize}

\section{おわりに}

この原稿では京都女子大学現代社会学部宮下ゼミの学生向けに,卒業論文のテンプレートを示した.
すばらしい卒業論文が作成されることを期待している.
このテンプレートに足りないところや誤りがあればいつでも指摘してほしい.

\section*{謝辞}

研究を進めたり卒論を作成したりしたときにお世話になった人(指導教員,友人,知人,親戚等)がいれば,この節に具体的に記して謝意を述べる.
この節の文章は敬体で良い.
例えば「○○教授には卒業論文作成全般にわたりご指導いただきました.深く感謝します.」
「○○氏には○○に関する重要な示唆をいただきました.感謝します.」
「○○についての有用なデータをご提供いただいた○○氏に感謝します.」など.

\begin{thebibliography}{99}
 \bibitem{newsweek} グレン・カール,元CIAスパイに学ぶ最高のライティング作法,Newsweek日本版,Vol.36,No.41,2021年10月26日号,pp.18--21,2021.
 \bibitem{refwww} Overleaf, \url{https://ja.overleaf.com/}, 2022年10月10日アクセス.
 \bibitem{okumura2020} 奥村晴彦, [改訂第8版]\LaTeX2e 美文書作成入門, 技術評論社, 2020.
 \bibitem{refjournal1} 宮下健輔,水野義之,京都女子大学における全学情報教育とそれを支える情報システムの変遷に関する考察,情報処理学会論文誌,Vol.53,No.3,pp. 997--1004,2012.
 \bibitem{refjournal2} Kensuke Miyashita, Yuki Maruno, Campus Wireless LAN Usage Analysis and Its Applications, NEURAL INFORMATION PROCESSING, ICONIP 2016, PT I, pp. 563--569, 2016.
 \bibitem{refbook1} 河野哲也,レポート・論文の書き方入門 第4版,慶応義塾大学出版会,2018.
 \bibitem{refbook2} 福地健太郎,園山隆輔,理工系のためのよい文章の書き方,翔泳社,2019.
\end{thebibliography}
\end{document}
